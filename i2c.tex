\chapter{I\textsuperscript{2}C sučelje mikrokontrolera STM32L471VGT6}
Za konfiguraciju kamere Arducam 5MP Mini Plus  PDH računalo koristi
I\textsuperscript{2}C komunikaciju. S obzirom na to da se za razvoj programske
potpore PDH računala koriste \textit{Low-Layer} biblioteke, potrebno je
razumijevanje načina rada I\textsuperscript{2}C periferije odabranog
mikrokontrolera kako bi se ispravno implementirali upravljački programi. U
nastavku slijedi općenit opis I\textsuperscript{2}C komunikacije kao i njena
implementacija na STM32L471VGT6 mikrokontroleru.

\section{I\textsuperscript{2}C protokol}
I\textsuperscript{2}C (\textit{Inter-Integrated Circuit}) je jednostavna
dvosmjerna sinkrona serijska sabirnica razvijena od strane \textit{Philips
Semiconductors} (sada \textit{NXP Semiconductors}) 1982. godine. Koristi dvije
linije:
\begin{itemize}
	\item serijska podatkovna linija (SDA, \textit{Serial Data Line}),
	\item serijska taktna linija (SCL, \textit{Serial Clock Line}),
\end{itemize}
obje linije su pritegnute na visoku logičku razinu preko \textit{pull-up}
otpornika. Moguće brzine prijenosa su:
\begin{itemize}
	\item do 100 \si{kbit/s} u \textit{Standard-mode} načinu rada, 
	\item do 400 \si{kbit/s} u \textit{Fast-mode} načinu rada,
	\item do 1 \si{Mbit/s} u \textit{Fast-mode Plus} načinu rada,
	\item do 3.4 \si{Mbit/s} u \textit{High-speed} načinu rada.
\end{itemize}
Navedene brzine se koriste kod dvosmjernog prijenosa, a moguća je i brzina do 5
\si{Mbit/s} u jednosmjernom prijenosu. Više uređaja se može spojiti na jednu
sabirnicu, a svaki uređaje je prepoznatljiv po svojoj jedinstvenoj adresi i
može se ponašati kao prijamnik ili odašiljač, ovisno o funkciji uređaja.
Protokol najčešće, a tako i u ovom slučaju, koristi 7-bitno adresiranje, a
moguće je i korištenje 10-bitnog adresiranja. Uz prijamnike i odašiljače uređaj
također može biti upravljač ili meta tijekom prijenosa podataka. Upravljač je
uređaj koji inicijalizira prijenos podataka na sabirnici i generira signal
takta kako bi omogućio prijenos. U tom trenutku, bilo koji uređaj koji je
adresiran smatra se metom.

Na I\textsuperscript{2}C sabirnicu se također može spojiti više upravljača, a
primjer jednog takvog spoja sa dva mikrokontrolera je dan na sljedećoj slici.
\begin{figure}[hp]
	\centering
	\includegraphics[width=\textwidth]{i2c_primjer_1.PNG}
	\caption{Primjer I\textsuperscript{2}C sabirnice sa spojena dva
	mikrokontrolera}
	\label{fig:i2c_primjer_1}
\end{figure}
Prijenos podataka bi možda mogao izgledati ovako:
\begin{enumerate}
	\item Mikrokontroler A želi poslati podatke mikrokontroleru B:
	\begin{itemize}
		\item mikrokontroler A (upravljač) adresira mikrokontroler B (meta)
		\item mikrokontroler A (upravljač-odašiljač) šalje podatke
		mikrokontroleru B (meta-prijamnik)
		\item mikrokontroler A prekida prijenos
	\end{itemize}
	\item Mikrokontroler A želi primiti podatke sa mikrokontrolera B:
		\begin{itemize}
		\item mikrokontroler A (upravljač) adresira mikrokontroler B (meta)
		\item mikrokontroler A (upravljač-prijamnik) prima podtke sa
		mikrokontrolera B (meta-odašiljač)
		\item mikroknotroler A prekida prijenos.
	\end{itemize}
\end{enumerate}
U svakom od navedenih slučajeva mikrokontroler A je generirao takt i prekidao
prijenos. Upravljač uvijek generira takt na I\textsuperscript{2}C sabirnici kod
prijenosa podataka. U ovom radu korišten je samo jedan mikrokontroler, odnosno
upravljač, pa ćemo se dalje usredotočiti samo na taj slučaj.

\subsection{Opis komunikacije i vremenski dijagram}
I\textsuperscript{2}C komunikacija započinje sa \textit{start} simbolom i
završava sa \textit{stop} simbolom. Komunikacijom se može čitati ili pisati
ovisno o R\textbackslash W bitu u adresi. Struktura adresiranja kod 7-bitne
adrese izgleda ovako:
\begin{center}
	\begin{tabular}{ | c | c | c | c | c | c | c | c | c | }
		\hline
		& \multicolumn{7}{|c|}{Adresno polje} & R\textbackslash W \\
		\hline
		Pozicija bita u bajtu & 7 & 6 & 5 & 4 & 3 & 2 & 1 & 0 \\
		\hline
		Značenje & MSB & \multicolumn{5}{|c|}{} & LSB & 1=READ, 0=WRITE \\
		\hline
	\end{tabular}
\end{center}
kao što se vidi, najmanje značajan bit označava želi li se nešto čitati ili
pisati.

Imajući na umu izgled adresnog bajta, vremenski dijagram tipčne
I\textsuperscript{2}C komunikacije izgleda ovako:
\begin{figure}[hp]
	\centering
	\includegraphics[width=\textwidth]{I2C_data_transfer.png}
	\caption{Vremenski dijagram I\textsuperscript{2}C komunikacije}
	\label{fig:i2c_timing_diagram}
\end{figure}
\begin{itemize}
	\item Prijenos podataka se inicijalizira \textit{start} uvjetom (S) tako da
	SDA linija prijeđe u nisku logičku razinu dok SCL linija ostaje u visokoj
	logičkoj razini.
	\item (Plavo područje) SCL prelazi u nisku logičku razinu i SDA postavlja
	prvi podatkovni bit dok je SCL u niskoj logičkoj razini.
	\item (Zeleno područje) Podaci se primaju dok SCL poraste za prvi bit
	(B\textsubscript{1}). Kako bi podaci bili valjani, SDA se ne smije
	promijeniti između rastućeg brida SCL-a i sljedećeg padajućeg brida.
	\item Postupak se ponavlja, SDA se postavlja dok je SCL u niskoj razini, a
	podaci se čitaju dok je SCL u visokoj razini (B\textsubscript{2} do
	B\textsubscript{n}).
	\item Nakon posljednjeg bita slijedi taktni impuls, tijekom kojeg SDA
	prelazi u nisku razinu pripremajući se za \textit{stop} uvjet.
	\item Signalizira se \textit{stop} uvjet kada SCL poraste, nakon čega
	slijedi porast SDA-a.
\end{itemize}
\textit{Start} i \textit{stop} uvjete uvijek generira upravljač.

Nakon svakog bajta prijamnik šalje odašiljaču ACK bit kojim se signalizira
uspješno primanje podatka, odnosno NACK bit kojim se signalizira neuspješno
primanje podatka. ACK i NACK bitovi se nazivaju signalom potvrde i definiraju
sljedeći način: odašiljač otpušta SDA liniju tijekom potvrdnog takta kako bi
prijamnik mogao spustiti SDA na nisku razinu na kojoj i ostaje tijekom visoke
razine takta. Ako SDA ostaje u visokoj razini tijekom devete periode takta, to
predstavlja NACK (engl. \textit{Not Acknowledge}) signal, a suprotan slučaj
predstavlja ACK (engl. \textit{Acknowledge}) signal. Ako je došlo do NACK
signala, upravljač može generirati \textit{stop} uvjet kako bi prekinuo prijenos
ili može ponovno generirati \textit{start} uvjet kako bi započeo nov prijenos.

Vremenski dijagram cijele komunikacije sa potvrdnim signalima prikazan je na
sljedećoj slici:
\begin{figure}[hp]
	\centering
	\includegraphics[width=\textwidth]{I2C_vremenski_dijagram.PNG}
	\caption{Prijenos podataka na I\textsuperscript{2}C sabirnici}
	\label{fig:i2c_timing_diagram_transaction}
\end{figure}

\section{Struktura I\textsuperscript{2}C periferije na STM32L471VGT6}