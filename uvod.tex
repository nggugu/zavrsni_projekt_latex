\chapter{Uvod}
Projekt FERSAT, koji se od 2018. godine provodi na Fakultetu elektrotehnike i računarstva, uključuje izradu, lansiranje i korištenje jednog nanosatelita CubeSat. Satelit u izradi dimenzija je približno 10 cm x 10 cm x 10 cm, volumena jedne litre i ne teži od 4/3 kilograma, što ga svrstava u skupinu satelita formata CubeSat 1U \cite{fersat_stranica_projekta}. Očekivani životni vijek satelita je 3 godine, a bit će postavljen u Zemljinoj orbiti na visini između 500 i 600 kilometara. Planirani korisni teret \engl{payload} FERSAT-a podijeljen je na tri podsustava:

    \begin{itemize}
        \item kamera za snimanje površine Zemlje i zemaljskog horizonta,
        \item detektori svjetla u vidljivom i ultraljubičastom dijelu spektra za mjerenje svjetlosnog onečišćenja i debljine stupca ozona,
        \item komunikacijski sustav u radijskom X-pojasu (10.45 GHz) za prijenos podataka na Zemlju.
    \end{itemize}

    Radom korisnog tereta upravlja \textit{Payload Data Handler} (PDH) računalo. Zadaća je PDH računala prikupiti podatke iz senzorskog podsustava i kamere, pohraniti ih u trajnu memoriju \engl{non-volatile memory} te poslati te podatke na Zemlju korištenjem komunikacijskog podsustava. Kao PDH računalo odabran je mikrokontroler STM32L471VGT6 proizvođača ST Microelectronics.

    Za rad ostalih podsustava satelita koji nisu direktno vezani uz koristan teret (npr. upravljanje položajem satelita, slanje telemetrijskih podataka na Zemlju) brine se \textit{Command and Data Handler} (CDH) računalo. CDH računalo također upravlja napajanjem korisnog tereta i šalje naredbe PDH računalu. Komunikacija CDH i PDH računala odvija se korištenjem sučelja CAN (\textit{Controller Area Network}). Konkretno CDH računalo u trenutku pisanja ovog teksta još nije odabrano.