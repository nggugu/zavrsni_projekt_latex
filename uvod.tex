\chapter{Uvod}

Ovaj završni projekt se izvodi u sklopu projekta FERSAT, koji se od 2018. godine provodi na Fakultetu elektrotehnike i računarstva Sveučilišta u Zagrebu \cite{fersat_stranica_projekta}. Cilj projekta je izrada, lansiranje i korištenje jednog nanosatelita u CubeSat formatu, dimenzija 10 cm x 10 cm x 10 cm, volumena jedne litre i težine ne veće od 4/3 kg. Navedene dimenzije satelita odgovaraju formatu CubeSat 1U. Planirana visina orbite satelita je između 500 i 600 km, a očekivano trajanje misije je 3 godine. Korisni teret satelita (engl. \textit{payload}) se sastoji od tri podsustava:

\begin{itemize}
	\item kamera za snimanje površine Zemlje i zemaljskog horizonta,
	\item detektori svjetla u vidljivom i ultraljubičastom dijelu spektra za mjerenje svjetlosnog onečišćenja i debljine stupca ozona,
	\item komunikacijski sustav u radijskom X-pojasu (10.45 GHz) za prijenos podataka na Zemlju.
\end{itemize}

Kako bi se moglo upravljati radom korisnog tereta, na satelit će biti ugrađeno PDH (engl. \textit{Payload Data Handler}) računalo, čija će zadaća biti prikupljanje podataka s kamere i senzorskog podsustava, pohranjivanje prikupljenih podataka u trajnu memoriju (engl. \textit{non-volatile memory}), te slanje podataka na Zemlju pomoću komunikacijskog sustava. Izabrani mikrokontroler za ulogu PDH računala je STM32L471VGT6 proizvođača ST Microelectronics.

Ostalim podsustavima, koji nisu direktno vezani uz koristan teret, upravlja CDH (engl. \textit{Command and Data Handler}) računalo. CDH računalo može upravljati položajem i orijentacijom satelita, slanjem telemetrijskih podataka na Zemlju, a također upravlja i napajanjem korisnog tereta i šalje naredbe PDH računalu preko CAN (engl. \textit{Controller Area Network}) sučelja. U trenutku pisanja ove dokumentacije, konkretno CDH računalo još nije odabrano.
    
Slika \ref{fig:fersat_blok} prikazuje blok dijagram cijelog sustava. U okviru ovog projekta razvijena je programska potpora PDH računala za upravljanje kamerom i \textit{flash} memorijom.

\begin{figure}[H]
	\centering
	\includegraphics[width=\textwidth]{fersat_blok_dijagram.png}
	\caption{Blok dijagram FERSAT-a i komunikacija sa zemaljskom postajom \cite{diplomski_goran_petrak}}
	\label{fig:fersat_blok}
\end{figure}

Sustav za upravljanje kamerom se sastoji od Arducam Mini 5MP Plus kamere. Upravljanje kamerom se sastoji od konfiguracije kamere i samog korištenja kamere, odnosno slikanja i spremanja slike. Konfiguracija kamere je nužna kako bi se ispravno podesili parametri trajanja ekspozicije, pojačanje i formata u kojem se slika želi spremiti.