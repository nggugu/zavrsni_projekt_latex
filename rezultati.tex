\chapter{Rezultati}

Komunikacija između osobnog računala i PDH sustava u ovom slučaju predstavlja komunikaciju između CDH računala i PDH sustava. Komunikacija se izvodi preko USART sučelja, te je stoga potreban emulator terminala koji može koristiti UART komunikaciju. U tu svrhu odabran je program CoolTerm, koji može primati i slati podatke preko UART komunikacije i može spremati rezultate komunikacije, što je potrebno kako bi se na računalu mogla učitati slika. UART komunikacija radi na brzini prijenosa podataka od 115200 i ona predstavlja slanje podataka preko XBand sustava.

Kada se sustav uključi prvo se ispisuje rezultat inicijalizacije sustava, ako je inicijalizacija dobro prošla ispisuje se poruka \verb|startup_ok|. Nakon toga nudi se izbor ponovne inicijalizacije datotečnog sustava, te se nudi izbor datoteke za slanje. Ako korisnik želi, može odmah nakon slanja datoteke odabranu datoteku izbrisati, a može i izbrisati datoteke bez prethodnog slanja. Korisnik može odabrati da se slanje datoteka preskoči. Sustav zatim traži od korisnika da koristi uobičajene, odnosne prethodne, postavke kamere ili da sam podesi postavke kamere ili da se korištenje kamere preskoči. Kada korisnik podesi sustav i odabere što želi raditi s njim, sustav ponudi korisniku pokretanje operacija PDH sustava. Kada sustav završi s izvršavanjem operacija ispisuje se rezultat operacija i status sustava, te sustav krene ispočetka i ponovno ponudi korisniku reinicijalizaciju datotečnog sustava. Primjer komunikacije između računala i PDH sustava dan je u isječku koda \ref{lst:default_comms}.

\begin{lstlisting}[caption=Komunikacija između računala i PDH sustava, label={lst:default_comms}]
sustav: startup_ok
sustav: Type 'reinit' to reinitialize the filesystem, 'n' not to
korisnik: n
sustav: Press 'y' to select file for transmission, 'n' to skip
korisnik: n
sustav: Any [other] file you would like to delete?['y'/'n']
korisnik: n
sustav: Press 'y' to set camera params, 'd' to use default/previous, 'n' to skip
korisnik: y
sustav: Set exposure nr_lines (uint16_t).
korisnik: 2000
sustav: Set exposure nr_lines frac (uint8_t).
korisnik: 10
sustav: Set gain (uint8_t), see acam.h.
korisnik: 9
sustav: Select format: 'r' for raw, 'j' for jpeg,
korisnik: j
sustav: Enter file name [0-1023] to store image measurments.
korisnik: 0
sustav: Press 's' to start PDH operations
korisni: s
sustav: Device camera status ok
sustav: Type 'reinit' to reinitialize the filesystem, 'n' not to
korisnik: n
sustav: Press 'y' to select file for transmission, 'n' to skip
korisnik: y
sustav: Enter file name [0-1023] of file to be transmitted via x-band.
korisnik: 0
sustav: Press 'y' to delete file after transmission, 'n' to keep it
korisnik: y
sustav: Any [other] file you would like to delete?['y'/'n']
korisnik: n
sustav: Press 'y' to set camera params, 'd' to use default/previous, 'n' to skip
korisnik: n
sustav: Press 's' to start PDH operations
korisnik: s
sustav: ÿØÿ
...
slanje slike
...
ÿÙ Device xband status ok
sustav: Type 'reinit' to reinitialize the filesystem, 'n' not to
\end{lstlisting}

Ovisno o svjetlini okoline gdje se nalazi kamera preporučuje se vrijednost parametra \verb|nr_lines| 2000 za relativno svjetlo područje i 5000 za relativno mračno područje. Za parametar \verb|nr_lines frac| se preporučuje mala vrijednost, na primjer 10. Isto tako, male vrijednosti se preporučuju za parametar \verb|gain|. Primjer slika koje su uslikane s navedenim parametrima u jpeg formatu prikazane su na slikama \ref{fig:dark} i \ref{fig:light}.
\begin{figure}[H]
	\centering
	\includegraphics[height=5 cm, angle=180]{light.jpeg}
	\caption{Slika ZESOI-knjižnice u relativno svjetlim uvjetima}
	\label{fig:light}
\end{figure}
\begin{figure}[H]
	\centering
	\includegraphics[height=5 cm, angle=90]{dark.jpeg}
	\caption{Slika narančastog autića u relativno mračnim uvjetima}
	\label{fig:dark}
\end{figure}