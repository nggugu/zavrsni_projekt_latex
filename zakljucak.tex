\chapter{Zaključak}

U ovom projektu opisan je razvoj programske podrške za upravljanje kamerom i \textit{flash} memorijom nanosatelita CubeSat u sklopu projekta FERSAT. Kratko su opisani FERSAT projekt i dosadašnje aktivnosti u sklopu projekta. Opisana je arhitektura korisnog tereta FERSAT satelita i sklopovlje PDH sustava. Detaljno su objašnjeni I\textsuperscript{2}C, SPI i CAN protokoli i objašnjena je razlika između I\textsuperscript{2}C i SPI periferijskih sklopova prethodno korištenog (STM32F407VGT6) i novog (STM32L471VGT6) mikrokontrolera. Dan je pregled prilagodbe postojeće programske podrške za stari mikrokontroler na novi mikrokontroler i objašnjene su poteškoće do kojih je došlo kod prilagodbe. Objašnjen je način korištenja CAN periferije i dan je pregled funkcija koje će se koristiti za CAN komunikaciju. Programska potpora za upravljanje kamerom i \textit{flash} memorijom je integrira u operacijski sustav za rad u stvarnom vremenu FreeRTOS. Predstavljeni su zadaci razvijeni u FreeRTOS operacijskom sustavu kojima se ostvaruje višezadaćnost PDH sustava i objašnjena je veza između zadataka. Objašnjen je način učitavanja razvijene programske podrške na PDH sustav. Predstavljeni su rezultati projekta i objašnjen je način upotrebe sustava, te su dane preporučene postavke za korištenje sustava.

Razvijenu programsku podršku moguće je nadograditi i poboljšati. Za primjer se uzima brzina takta SPI komunikacije između. Trenutačne brzine takta su male (1.25 Mbit/s za kameru i 625 kbit/s za \textit{flash} memoriju) jer je kod većih brzina primijećeno izobličenje takta, što može stvarati probleme u komunikaciji. Moglo bi se iterativno povećavati brzina takta dok se ne dobije dovoljno velika brzina uz minimalno izobličenje signala takta.

Radi pogrešaka na tiskanoj pločici PDH sustava neke dijelove programske podrške nije bilo moguće razviti. Na primjer podsustav za CAN komunikaciju sadržava krivo spojene stezaljke za napajanje, a pogreške se nalaze i na samom podsustavu za napajanje PDH sustava, pa projekt u trenutačnom stanju ne može raditi bez osobnog računala i programatora. Kako bi se riješili ti problemi potrebno je napraviti novu ispravnu tiskanu pločicu, što je van okvira ovog projekta.