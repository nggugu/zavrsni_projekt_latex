\chapter{Zaključak}

U ovom projektu opisan je razvoj programske podrške za 
upravljanje kamerom i \textit{flash} memorijom nanosatelita 
CubeSat u sklopu projekta FERSAT. Kratko su opisane i dosadašnje
 aktivnosti u sklopu projekta FERSAT. Opisana je arhitektura korisnog
 tereta FERSAT satelita i sklopovlje PDH sustava. 
 Detaljno su objašnjeni I\textsuperscript{2}C, SPI i 
 CAN protokoli i objašnjena je razlika između 
 I\textsuperscript{2}C i SPI periferijskih sklopova prethodno 
 korištenog (STM32F407VGT6) i trenutno korištenog (STM32L471VGT6) 
 mikrokontrolera. Dan je pregled prilagodbe postojeće programske podrške na novi mikrokontroler i 
 objašnjene su poteškoće do kojih je došlo kod prilagodbe. Objašnjen je način korištenja CAN periferije i dan je pregled 
 funkcija koje će se koristiti za CAN komunikaciju. 
 Programska potpora za upravljanje kamerom i 
 \textit{flash} memorijom je integrirana u operacijski sustav 
 za rad u stvarnom vremenu FreeRTOS. 
 Predstavljeni su zadaci razvijeni u FreeRTOS operacijskom 
 sustavu kojima se ostvaruje višezadaćnost PDH sustava i 
 objašnjena je veza između zadataka. Objašnjen je način 
 učitavanja razvijene programske podrške na PDH sustav. 
 Predstavljeni su rezultati projekta i objašnjen je način 
 upotrebe sustava, te su dane preporučene postavke za 
 korištenje sustava.

Razvijenu programsku potporu moguće je nadograditi i poboljšati. 
Za primjer se uzima brzina takta SPI komunikacije između \textit{flash} memorije i mikrokontrolera, i kamere i mikrokontrolera. % TODO tu nesto fali
 Trenutačne brzine prijenosa su male (1,25 Mbit/s za kameru i 625 kbit/s za \textit{flash} memoriju) 
 jer je kod većih brzina primijećeno izobličenje signala takta, što može stvarati probleme u komunikaciji. 
U okviru budućeg rada moglo bi se ispitati ograničenja na maksimalnu 
brzinu prijenosa sginala s kamere.

Radi pogrešaka na tiskanoj pločici PDH sustava neke dijelove programske podrške nije bilo moguće razviti. 
Primjerice, na podsustavu za CAN komunikaciju priključci
za napajanje su pogrešno spojeni, a pogreške se nalaze 
i na samom podsustavu za napajanje PDH sustava, 
pa projekt u trenutačnom stanju ne može raditi bez 
osobnog računala i programatora. 
Kako bi se riješili ti problemi potrebno je napraviti novu 
ispravnu tiskanu pločicu, što je van okvira ovog projekta.
